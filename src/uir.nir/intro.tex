\chapter*{Введение}\addcontentsline{toc}{chapter}{Введение}
В современном мире электронная почта - далеко не самый
распространенный способ общения в сети. Несмотря на это,
наряду с сотовой sсвязью, она остается важнейшим инструментом коммуникации.
Количество писем растет, множатся ящики электронной почты, а времени на поиск действительно важного письма
среди многочисленного спама у современного человека все меньше. По данным Лаборатории Касперского за 2020
год, доля спама в почтовом трафике превышает 50\%, причем большая его часть (21,27\%) пришла из России. \cite{stat}
На этом фоне более быстрое и точное детектирование спама остается актуальной задачей.

Для решения задачи фильтрации нежелательных рассылок совместно с основными методами широко применяются
природные алгоритмы оптимизации. В течение сотен лет благодаря действию эволюционных механизмов многие
живые организмы развили особые способности, не дающие им погибнуть. Этот успех биологических организмов
вдохновил всевозможных исследователей,  занимающихся задачами оптимизации, на создание так называемых
природных алгоритмов. За последние годы сообщество таких исследователей значительно выросло, достигнув
большого разнообразия в том, что касается их источников вдохновения. \cite{BioInspiredTaxonomy}

Современными исследованиями в области природных алгоритмов занимаются такие учёные, как доктор Peter J. Bentley,
Derviş Karaboğa, Xin-She Yang. Существует множество подходов к фильтрации спама, использующих алгоритмы машинного
обучения. В данной работе природные алгоритмы будут применяться к полиномиальному наивному байесовскому классификатору,
показавшему лучшие резулsьтаты в совместном использовании с генетическим алгоритмом и методом роя частиц. \cite{IEEE}

Данная работа, в соответствии с федеральным государственным образовательным стандартом высшего профессионального
образования по направлению подготовки 090900 «Информационная безопасность», отвечает следующим задачам
экспериментально-исследовательской деятельности: сбор, изучение научно-технической информации, отечественного и
зарубежного опыта по тематике исследования; проведение экспериментов по заданной методике, обработка и анализ результатов.

Все представленные в работе результаты получены автором лично.

В первом разделе проводится обзор алгоритмов машинного обучения, используемых для обнаружения спама.

Во втором разделе проводится обзор возможностей природных алгоритмов.

В третьем разделе производится разработка метода применения природных алгоритмов для повышения эффективности
полиномиального наивного байесовского классификатора. Кроме того, производится подготовка датасетов, необходимых для тестирования вышеуказанных алгоритмов и проведения эксперимента.

В четвертом разделе рассмотрены вопросы, касающиеся организации разработки и выбора инструментальных средств.
Также представлено созданное вспомогательное программное обеспечение для проведения сравнительного анализа
метрик машинного обучения для полученных модификаций.

В пятом разделе описан процесс и результаты тестирования модификаций полиномиального наивного байесовского
классификатора.

