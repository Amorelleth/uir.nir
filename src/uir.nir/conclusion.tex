\chapter*{Заключение}\addcontentsline{toc}{chapter}{Заключение}

В рамках данной работы осуществлен обзор различных подходов к фильтрации
спама. В частности, задача фильтрация спама была рассмотрена как задача
классификации в контексте машинного обучения. Описаны некоторые алгоритмы
машинного обучения, используемые для решения данной задачи. Кроме того,
приведена классификация природных алгоритмов.
Также приведены различные метрики, используемые для оценки производительности
моделей машинного обучения. Перечислены основные подходы к оптимизации гиперпараметров
машинного обучения.

В ходе работы получены следующие результаты:
\begin{enumerate}

    \item[—] Спланирован эксперимент по выявлению достоинств и недостатков различных
        подходов к оптимизации классификаторов спама на основе машинного обучения;

    \item[—] Спроектирована система для проведения эксперимента по выявлению достоинств и
        недостатков различных подходов к оптимизации классификаторов спама на основе машинного
        обучения;

    \item[—] Для реализации программной системы выбран язык Python версии 3.9 или выше,
        так как он предоставляет широкий выбор библиотек для машинного обучения, в частности,
        библиотеку scikit-learn. Таким образом, его использование призвано облегчить программную
        реализацию алгоритмов машинного обучения и сосредоточиться непосредственно на
        эксперименте.
\end{enumerate}

В дальнейших исследованиях планируется практическая реализация представленной системы,
проведение эксперимента в соответствии с разработанным планом с последующим
выявлением достоинств и недостатков различных подходов к оптимизации гиперпараметров
классификаторов спама.
