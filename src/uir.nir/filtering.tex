\chapter*{Классификация методов фильтрации спама в электронной почте}\label{Chapter:Filtering}

Одна из причин, по которой спам трудно фильтровать, заключается в
его динамическом характере. Характеристики спама, такие как темы,
повторяющиеся термины и т. д., в электронной почте быстро меняются
со временем, поскольку появляются новые способы обхода спам-фильтров.
Эти способы включают: обфускацию слов, спам в виде изображений,
рассылку спама по электронной почте с взломанных компьютеров и другие.
Понимание природы и эволюции спама может помочь в разработке мер
противодействия. Нет универсального метода защиты от спама, однако
наилучший подход должен обладать механизмом для определения эволюции
характеристик спама. Среди всех традиционных подходов огромного успеха
в борьбе со спамом удалось достичь благодаря фильтрации на основе
содержимого, в частности, системам на основе машинного обучения.
Они обучаются и адаптируются к новым угрозам, реагируя на меры
противодействия спамеров.

Классификация методов фильтрации спама в электронной почте,
согласно работе \cite{filters}, включает в себя следующие подходы:
\begin{enumerate}
    \item На основе репутации (Reputation-based)
        \begin{enumerate}
            \item Анализ источника сообщения
            \begin{enumerate}
                \item Blacklisting — создание и поддержание на уровне пользователя 
                или сервера списка адресов электронной почты или IP-адресов сервера, 
                с которого, как установлено, исходит спам. Сообщение автоматически 
                блокируется на этапе подключения SMTP.
                \item Whitelisting — в противоположность черному списку, список 
                предварительно утвержденных или доверенных контактов, доменов или 
                IP-адресов, которые могут общаться с пользователем почты.
                \item Анализ разнообразия отправителей.
            \end{enumerate}
        \item Анализ социальных сетей — социальные сети очень полезны для определения 
        надежности отправителей, поэтому в подходах к фильтрации спама стали использоваться 
        взаимодействия в социальных сетях. Например, авторы [4] анализировали поля заголовка 
        электронной почты, чтобы построить граф социальной сети пользователя, а затем классифицировали 
        сообщения электронной почты на основе «коэффициента кластеризации» подкомпонента графа.
        \item Анализ трафика
        \item Анализ протокола
    \end{enumerate}
\end{enumerate}
   