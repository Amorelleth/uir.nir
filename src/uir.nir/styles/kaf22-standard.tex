\setlength\paperheight{297mm}
\setlength\paperwidth{210mm}


\usepackage{polyglossia}
\setmainlanguage[numerals=cyrillic]{russian}
\setotherlanguages{english}

\usepackage{xunicode} % some extra unicode support
%\usepackage[utf8x]{inputenc}
\usepackage{xltxtra} % \XeLaTeX macro
\usepackage{fontspec}
\defaultfontfeatures{Ligatures=TeX}

\newfontfamily{\cyrillicfont}{Times New Roman}
\setmainfont[Mapping=tex-text]{Times New Roman}
\newfontfamily{\cyrillicfonttt}{Courier New}
\setmonofont{Courier New}

% \setmainfont{Linux Libertine O}
% \setsansfont{Linux Biolinum O}
% \setmonofont[SmallCapsFont={Latin Modern Mono Caps}]{Latin Modern Mono Light}

%нумерация справа и колонтитулы справа вверху
\usepackage{fancyhdr}
\usepackage[left=30mm,right=10mm,top=20mm,bottom=20mm,bindingoffset=0cm]{geometry}%

\usepackage{amsfonts}
\usepackage{amssymb}
\usepackage{amsmath}
\usepackage{amsthm}

\usepackage{calc}
\usepackage{ifthen}
\usepackage{graphicx}
\usepackage{array}
\usepackage{pdfpages}
\usepackage{longtable}
\usepackage{tabu}
\usepackage{indentfirst}
\usepackage{float}
\usepackage{graphicx}
\newcolumntype{P}[1]{>{\linespread{1.5}}{p{#1}}

%Точка после номера раздела
\usepackage{misccorr}

\usepackage[unicode=true]{hyperref}
\usepackage{color}
\usepackage{pgf}

\usepackage{./styles/pstheorems}

\usepackage{lipsum}
\usepackage{enumitem}
\setlength{\parindent}{12.5mm}
\setlength{\baselineskip}{1.5em}
\setlist[enumerate]{wide,topsep=0pt,itemsep=0pt,partopsep=0pt,parsep=0pt}
\setlist[itemize]{wide,topsep=0pt,itemsep=0pt,labelsep=7.5mm,partopsep=0pt,parsep=0pt}
% \setlist[itemize]{leftmargin=*}
% \setlist[enumrate]{leftmargin=*}

% Настройка списков (без лишних вертикальных отступов)
% \usepackage{paralist}
% \setdefaultenum{1.}{1.}{1.}{1.}
% % \setdefaultitem{{}--{}}{}{}{}
% \setdefaultleftmargin{2cm}{2cm}{}{}{}{}
% % \setlength\itemsep{-1em}
% \let\itemize\compactitem
% \let\enditemize\endcompactitem
% \let\enumerate\compactenum
% \let\endenumerate\endcompactenum
% \let\description\compactdesc
% \let\enddescription\endcompactdesc
% \pltopsep=\smallskipamount
% \plitemsep=0pt
% \plparsep=0pt
% Команда для отмены разрыва страниц перед списками
\makeatletter
\newcommand\mynobreakpar{\par\nobreak\@afterheading}
\makeatother
%%%%%%



\usepackage[singlelinecheck=false,labelsep=endash]{caption}
\captionsetup[table]{justification=justified}
\captionsetup[figure]{justification=centering}

\usepackage{titlesec}
\titleformat{\chapter}[block]{\centering\normalfont\bfseries}{\MakeUppercase{\chaptertitlename}\ \thechapter}{1ex}{\MakeUppercase}
\titlespacing{\chapter}{0pt}{\parskip}{-\parskip}%{0em}{2em}

\titleformat{\section}[block]{\normalfont\bfseries}{\thesection}{1ex}{}{}
\titlespacing{\section}{12.5mm}{\parskip}{-\parskip}%{0em}{1ex}

\titleformat{\subsection}[block]{\normalfont\bfseries}{\thesubsection}{1ex}{}{}
\titlespacing{\subsection}{12.5mm}{\parskip}{-\parskip}%{0em}{1ex}

\titleformat{\subsubsection}[block]{\normalfont\bfseries}{\thesubsubsection}{1ex}{}{}
\titlespacing{\subsubsection}{12.5mm}{\parskip}{-\parskip}%{0em}{1ex}
% paragraph и subparagraph -- в тексте, без отступов

\setcounter{secnumdepth}{4}

\titleformat{\paragraph}[block]{\normalfont\normalsize\bfseries}{\theparagraph}{1ex}{}{}
\titlespacing{\paragraph}{12.5mm}{\parskip}{-\parskip}%{0em}{1ex}


\titleformat{\subparagraph}[runin]{\normalfont\normalsize\bfseries}{\thesubparagraph}{0pt}{}{}
\titlespacing{\subparagraph}{0pt}{0em}{0ex}


% Своё название для Cписка литературы
\usepackage[title, titletoc]{appendix}
\addto\captionsrussian{% Replace "english" with the language you use
  \renewcommand{\contentsname}%
  {Содержание}%
}

\usepackage{./styles/mathpartir}

\makeatletter
\let\ps@plain\ps@fancy              % Подчиняем первые страницы каждой главы общим правилам
\makeatother
\pagestyle{fancy}
\fancyhf{}
\fancyfoot[C]{\thepage}

\renewcommand{\thesection}{\arabic{section}}
\renewcommand{\theparagraph}{\arabic{paragraph}}
\renewcommand{\headrulewidth}{0pt}
\renewcommand{\footrulewidth}{0pt}
\renewcommand{\baselinestretch}{1.5}
\newcommand{\headertext}[1]{\fancyhead[R]{\tiny{#1}}}

%% Список литературы

\makeatletter
\bibliographystyle{ugost2008}     % Оформляем список литературы по ГОСТ 7.1
% (ГОСТ Р 7.0.11-2011, 5.6.7)
\renewcommand{\@biblabel}[1]{#1.}   % Заменяем список литературы с квадратных
% скобок на точку
\makeatother

%\frenchspacing %% изменение расстояние до и после точек в ряде случаев

\renewcommand{\theenumi}{\arabic{enumi}}
\renewcommand{\theenumii}{\arabic{enumii}}
\newcommand{\theenumiii}{\arabic{enumiii}}
\renewcommand{\theenumiv}{\arabic{enumiv}}

\renewcommand{\labelenumi}{\theenumi.}
\renewcommand{\labelenumii}{\theenumi.\theenumii.}
\renewcommand{\labelenumiii}{\theenumi.\theenumii.\theenumiii.}
\renewcommand{\labelenumiv}{\theenumi.\theenumii.\theenumiii.\theenumiv.}

\providecommand{\showannotation}{false}
\newenvironment{annotation}{}{}
\usepackage{ifthen}
\usepackage{environ}
\ifthenelse{\equal{\showannotation}{false}}{
  \RenewEnviron{annotation}{}
}{}

\usepackage{listingsutf8}

\renewcommand{\lstlistingname}{Листинг}

\lstset{
language=[Sharp]C,
basicstyle=\linespread{0.94}\ttfamily,
tabsize=2,
showstringspaces=false,
columns=flexible,
numbers=left,
numberstyle=\normalsize\color{gray},
breaklines=true,
breakatwhitespace=true,
framesep=6pt,
abovecaptionskip=1em,
captionpos=b,
extendedchars=\true,
inputencoding=utf8,
xleftmargin=22pt,
literate={Ö}{{\"O}}1
{Ä}{{\"A}}1
{Ü}{{\"U}}1
{ß}{{\ss}}1
{ü}{{\"u}}1
{ä}{{\"a}}1
{ö}{{\"o}}1
{~}{{\textasciitilde}}1
{а}{{\selectfont\char224}}1
{б}{{\selectfont\char225}}1
{в}{{\selectfont\char226}}1
{г}{{\selectfont\char227}}1
{д}{{\selectfont\char228}}1
{е}{{\selectfont\char229}}1
{ё}{{\"e}}1
{ж}{{\selectfont\char230}}1
{з}{{\selectfont\char231}}1
{и}{{\selectfont\char232}}1
{й}{{\selectfont\char233}}1
{к}{{\selectfont\char234}}1
{л}{{\selectfont\char235}}1
{м}{{\selectfont\char236}}1
{н}{{\selectfont\char237}}1
{о}{{\selectfont\char238}}1
{п}{{\selectfont\char239}}1
{р}{{\selectfont\char240}}1
{с}{{\selectfont\char241}}1
{т}{{\selectfont\char242}}1
{у}{{\selectfont\char243}}1
{ф}{{\selectfont\char244}}1
{х}{{\selectfont\char245}}1
{ц}{{\selectfont\char246}}1
{ч}{{\selectfont\char247}}1
{ш}{{\selectfont\char248}}1
{щ}{{\selectfont\char249}}1
{ъ}{{\selectfont\char250}}1
{ы}{{\selectfont\char251}}1
{ь}{{\selectfont\char252}}1
{э}{{\selectfont\char253}}1
{ю}{{\selectfont\char254}}1
{я}{{\selectfont\char255}}1
{А}{{\selectfont\char192}}1
{Б}{{\selectfont\char193}}1
{В}{{\selectfont\char194}}1
{Г}{{\selectfont\char195}}1
{Д}{{\selectfont\char196}}1
{Е}{{\selectfont\char197}}1
{Ё}{{\"E}}1
{Ж}{{\selectfont\char198}}1
{З}{{\selectfont\char199}}1
{И}{{\selectfont\char200}}1
{Й}{{\selectfont\char201}}1
{К}{{\selectfont\char202}}1
{Л}{{\selectfont\char203}}1
{М}{{\selectfont\char204}}1
{Н}{{\selectfont\char205}}1
{О}{{\selectfont\char206}}1
{П}{{\selectfont\char207}}1
{Р}{{\selectfont\char208}}1
{С}{{\selectfont\char209}}1
{Т}{{\selectfont\char210}}1
{У}{{\selectfont\char211}}1
{Ф}{{\selectfont\char212}}1
{Х}{{\selectfont\char213}}1
{Ц}{{\selectfont\char214}}1
{Ч}{{\selectfont\char215}}1
{Ш}{{\selectfont\char216}}1
{Щ}{{\selectfont\char217}}1
{Ъ}{{\selectfont\char218}}1
{Ы}{{\selectfont\char219}}1
{Ь}{{\selectfont\char220}}1
{Э}{{\selectfont\char221}}1
{Ю}{{\selectfont\char222}}1
{Я}{{\selectfont\char223}}1
{…}{\ldots}1
{–}{-}1
{\ }{ }1
{-}{-}1
}


\headertext{}
\addto{\captionsrussian}{\renewcommand{\bibname}{Список использованных источников}}

\usepackage{lastpage}

\usepackage{graphicx}
\usepackage{setspace}

\newcommand{\specialcell}[2][c]{%
  \begin{tabular}[#1]{@{}c@{}}#2\end{tabular}}

\newcommand{\appendixitem}[1]{
  \refstepcounter{chapter}
  \chapter*{\appendixname\ \Asbuk{chapter}. #1}
  \addcontentsline{toc}{chapter}{\appendixname\ \Asbuk{chapter}\hspace{1ex}#1}
}
\renewcommand{\appendixname}{Приложение}
