\section{Обзор природных алгоритмов}\label{Section:Bio}

В последние годы в литературе описано большое количество алгоритмов,
основанных на природе и биологии. Это семейство алгоритмов симулирует
различные биологические процессы, наблюдаемые в природе, чтобы решать
сложные задачи оптимизации \cite{Yang2009}. Каждый естественный процесс
можно считать адаптируемым и имитируемым для создания нового
метаэвристического подхода, но с различными возможностями достижения
глобальных оптимальных решений для задач оптимизации \cite{BioInspiredTaxonomy}.

\subsection{Задача оптимизации}

Без преувеличения можно сказать, что в оптимизация используется повсюду: от
инженерного проектирования до бизнес-планирования и маршрутизации
Интернета до планирования праздников. Почти во всех этих действиях мы
пытаемся достичь определенных целей или оптимизировать что-то, например
прибыль, качество и время. Поскольку в реальных приложениях ресурсы,
время и деньги всегда ограничены, мы должны найти решения для оптимального
использования этих ценных ресурсов при различных ограничениях.
Математическая оптимизация или программирование - это изучение таких
проблем планирования и проектирования с использованием математических
инструментов. В настоящее время компьютерное моделирование становится
незаменимым инструментом для решения таких задач оптимизации с помощью
различных эффективных алгоритмов поиска.

С математической точки зрения, большинство задач оптимизации можно описать
в общем виде:

\begin{equation}\label{eq7}
    \underset{X \in R^n}{minimize} \: f_i(X), (i=1,2,\dots,M),
\end{equation}

\begin{equation}\label{eq8}
    subject\: to \: h_j(X) = 0, (j=1,2,\dots,J),
\end{equation}

\begin{equation}\label{eq9}
    g_k(X) \leq 0, (k=1,2,\dots,K)
\end{equation}

Где $f_i(X), h_j(X), g_k(X)$ — функции порождающего вектора

\begin{equation}\label{eq10}
    X = (x1,x2,\dots,x_n)^T
\end{equation}

Здесь компоненты $x_i$ переменной $x$ называются переменными решения.
Функции $f_i(x)$, где $i = 1, 2, \dots, M$, называются \emph{целевыми функциями}
или функциями стоимости. Пространство, охватываемое переменные решения,
называется \emph{пространством поиска} $R^n$, а пространство, образованное значениями
целевых функций, называются \emph{пространством решений} или пространством ответов.
Равенство \eqref{eq8} с $h_j$ и неравенство \eqref{eq9} с $g_k$ называются \emph{ограничениями}.

Проблемы оптимизации обычно описывают в терминах локальной или глобальной
оптимизации:

\begin{itemize}
    \item[—]
        \emph{Локальная оптимизация}, при которой алгоритм ищет точку, которая является
        только локально оптимальной, что означает, что она минимизирует целевую
        функцию среди возможных точек, которые находятся рядом с ней. Методы локальной оптимизации широко используются в приложениях, где есть смысл
        найти хорошую, если не самую лучшую точку \cite{Boyd2004};

    \item[—]
        \emph{Глобальная оптимизация}, при которой алгоритм ищет глобальный оптимум,
        используя механизмы для поиска в более крупных частях пространства поиска.
        Глобальная оптимизация используется для задач с небольшим количеством
        переменных, где время вычислений не критично, а ценность поиска истинного
        глобального решения очень высока \cite{Boyd2004}.
\end{itemize}

\subsection{Оптимизация гиперпараметров в алгоритмах МО}\label{optimization}

Гиперпараметр модели — это параметр, который является внешним по
отношению к модели и значение которого невозможно оценить по данным.
Мы не можем знать наилучшее значение гиперпараметра модели для той или иной проблемы.
Часто гиперпараметры можно установить с помощью эвристики.

К основным подходам к оптимизации гиперпараметров относятся:

\begin{itemize}
    \item[—] \emph{Поиск по сетке} представляет собой простой перебор
        по заданному вручную подмножеству гиперпараметрического пространства
        алгоритма обучения;
    \item[—] \emph{Случайный поиск} заменяет перебор всех комбинаций их случайным выбором;
    \item[—] \emph{Градиентный спуск} — для конкретных алгоритмов МО (например, машины опорных векторов)
        можно вычислить градиент относительно гиперпараметров, а затем
        оптимизировать гиперпараметры с помощью градиентного спуска;
    \item[—] Идея \emph{байесовской оптимизации} состоит в построении
        вероятностной модели целевой функции и использовании ее для выбора
        наиболее подходящих гиперпараметров;
    \item[—] \emph{Эвристический подход} подразумевает применение различных эвристическикх и
        метаэвристических алгоритмов оптимизации.
\end{itemize}

\subsection{Алгоритмы оптимизации}

Также алгоритмы оптимизации можно разделить на \emph{детерминированные} и
\emph{стохастические}. В отличие от детерминированных моделей,
которые дают одинаковые точные результаты для определенного набора
входных данных, стохастические модели представляют данные и
предсказывают результаты, учитывающие определенные уровни
непредсказуемости или случайности.

Стохастические алгоритмы бывают \emph{эвристические} и \emph{метаэвристические}.
Можно сказать, что эвристика использует зависящую от проблемы информацию
для поиска «достаточно хорошего» решения конкретной проблемы, в
то время как метаэвристика представляет собой общие алгоритмические
идеи, которые можно применять к широкому кругу проблем.

Стоит отметить, что в литературе не существует согласованных определений
эвристики и метаэвристики. Некоторые используют термины «эвристика» и
«метаэвристика» как синонимы. Однако в последнее время все стохастические
алгоритмы с рандомизацией и локальным поиском называются
метаэвристическими \cite{Yang2009}.

\subsection{Классификация природных алгоритмов}

Классифицировать природные алгоритмы можно по разным признакам, однако наиболее
часто этим признаком служит биологический источник вдохновения \cite{BioInspiredTaxonomy}.

\subsubsection{Эволюционные алгоритмы}

В эту категорию входят алгоритмы, основанные на популяциях, а также
на принципах естественной эволюции. К ним относятся как
классические алгоритмы эволюционных вычислений, такие как генетический
алгоритм (GA), стратегии эволюции (ES) и дифференциальная эволюция (DE),
таки менее известные алгоритмы, основанные на воспроизведении
различных биологических организмов, таких как пчелиные матки и сорняки.

\subsubsection{Алгоритмы на основе роевого интеллекта}

Роевой интеллект (SI) — это уже устоявшийся термин. Его можно определить
как коллективное поведение децентрализованных, самоорганизованных систем в
естественной или искусственной среде. Выражение было предложено в контексте
роботизированных систем, но с годами обобщено, чтобы обозначить возникновение
коллективного разума из группы простых агентов, управляемых простыми поведенческими
правилами. Таким образом, биологические метаэвристики, относящиеся к роевому
интеллекту, основаны на коллективном поведении сообществ животных, таких как
колонии насекомых или стаи птиц, в которых коллективный разум позволяет решать
задачи оптимизации.

Разделение на подкатегории по среде может быть следующим:

\begin{itemize}
    \item[—] Летающие животные

        Эта категория включает в себя метаэвристику, основанную на
        концепции роевого интеллекта, в которой траектория агентов
        определяется полетом, как у птиц, летучих мышей или других
        летающих животных. Самыми известными алгоритмами в этой
        подкатегории являются метод роя частиц (PSO) [2] и алгоритм пчелиной
        колонии (ABC);

    \item[—] Наземные животные

        Метаэвристика в этой категории основана на поисках пищи или
        передвижениях наземных животных. Наиболее известным подходом в
        этой категории является муравьиный алгоритм (ACO),
        который воспроизводит механизм, используемый муравьями для
        определения местоположения источников пищи и
        информирования об их существовании своим собратьям в колонии.
        В эту категорию также входят другие популярные алгоритмы, такие
        как алгоритм стаи серых волков (GWO), алгоритм льва (LOA),
        которые имитируют методы охоты, используемые этими животными;

    \item[—] Водные животные

        Этот тип метаэвристических алгоритмов основан на поведении водных животных.
        Сюда относятся такие популярные алгоритмы, такие как
        стадо криля (KH), алгоритм китов (WOA), а также алгоритмы,
        основанные на эхолокации, используемой дельфинами для обнаружения рыб,
        такие как оптимизация партнеров дельфинов (DPO) и эхолокация дельфинов (DEO);

    \item[—] Микроорганизмы

        Метаэвристика, основанная на микроорганизмах, связана с поиском пищи
        бактериями. Колония бактерий совершает движение в поисках пищи. Другие
        типы метаэвристик, которые могут быть частью этой категории,
        связаны с вирусами, которые обычно воспроизводят процесс заражения клетки
        вирусом. Наиболее известным алгоритмом этой категории является алгоритм
        оптимизации сбора бактерий (BFOA).

\end{itemize}


При разделении на подкатегории по вдохновляющему поведению каждый
алгоритм классифицируется как принадлежащий к одному из следующих
поведенческих паттернов:

\begin{itemize}
    \item[—] Движение

        Алгоритм относится к подкатегории
        движения, если биологическое вдохновение в основном состоит в
        том, как животное перемещается в окружающей среде;

    \item[—] Собирательство

        В алгоритмах на основе собирательства поведение животного определяет
        механизм, используемый для получения пищи.

\end{itemize}


\subsubsection{Алгоритмы на основе физики / химии}

Алгоритмы этой категории характеризуются тем, что они имитируют поведение
физических или химических явлений, таких как гравитационные силы,
электромагнетизм, электрические заряды и движение воды, а также химические
реакции и движение частиц газа. В этой категории мы можем найти
некоторые хорошо известные алгоритмы, разработанные в прошлом веке,
такие как алгоритм имитации отжига (SA) или алгоритм гравитационного поиска (GSA).
Другие алгоритмы, такие как гармонический поиск (HS), относятся к процессу сочинения
музыки, человеческому изобретению, которое имеет больше общего с другими
физическими алгоритмами в том, что касается использования звуковых волн,
чем с алгоритмами, основанными на социальном поведении человека.

\subsubsection{Алгоритмы, основанные на социальном поведении человека}

Алгоритмы, попадающие в эту категорию, вдохновлены человеческими социальными
концепциями, такими как принятие решений и идеями, связанными с конкуренцией
идеологий внутри общества, таких как алгоритм идеологии (IA) или политическими
концепциями, такими как алгоритм империалистической колонии (ICA). В эту
категорию также входят алгоритмы, имитирующие спортивные соревнования,
такие как алгоритм футбольной лиги (SLC). Процессы мозгового штурма также заложили
вдохновляющие основы нескольких алгоритмов, таких как алгоритм оптимизации
мозгового штурма (BSO.2) и глобальный передовой алгоритм оптимизации мозгового
штурма (GBSO).

\subsubsection{Алгоритмы на основе растений}

Эта категория объединяет все алгоритмы оптимизации, поиск которых
основан на растениях. В этом случае, в отличие от методов в категории
роевого интеллекта, нет связи между агентами. Один из самых известных — это
алгоритм оптимизации леса (FOA.1), вдохновленный процессом
воспроизводства растений.

\subsubsection{Алгоритмы со смешанными источниками вдохновения}

В эту категорию входят алгоритмы, которые не подходят ни к одной из
предыдущих, то есть в ней можно обнаружить алгоритмы с различными
характеристиками, такими как оптимизация пар Инь-Ян (YYOP).
Хотя эта категория неоднородна и не демонстрирует
единообразия среди алгоритмов, которые представляет, ее включение
в классификацию служит примером очень разных источников вдохновения,
существующих в литературе. С появлением новых алгоритмов эта группа
должна дать начало новым категориям.

