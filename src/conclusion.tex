\chapter*{Заключение}\addcontentsline{toc}{chapter}{Заключение}

Фильтрация спама была рассмотрена в контексте машинного обучения как задача
классификации. Описаны некоторые алгоритмы машинного обучения, используемые 
для решения данной задачи. Кроме того, приведена классификация природных алгоритмов 
по биологическому источнику вдохновения.
Также приведены метрики, используемые для оценки производительности
моделей машинного обучения, перечислены основные подходы к решению задачи оптимизации параметров
обучающих алгоритмов. В рамках данной задачи спланирован и проведен эксперимент по сравнению 
эффективности нескольких природных алгоритмов.

В ходе работы получены следующие результаты:
\begin{itemize}
    \item[—] Спроектирован программный комплекс для проведения эксперимента по сравнению эффективности 
        выбранных алгоритмов в задаче оптимизации классификатора спама на основе машинного обучения;
    \item[—] Выбраны алгоритмы для эксперимента — оптимизатор орла (AO), поиск голодных игр (HGS), алгоритм поиска 
        воробьев (SSA) и алгоритм оптимизации кормодобывания скатов манта (MRFO);
    \item[—] Выбраны метрики для оценки эффективности природных алгоритмов в 
        вышеуказанной задаче — $Acuracy$ и $F1-score$; 
    \item[—] Создан программный комплекс для проведения эксперимента;
    \item[—] Проведен эксперимент, на основе выбранных метрик сделаны выводы об 
    эффективности выбранных алгоритмов:
        \begin{itemize}
            \item[а] Все выбранные алгоритмы, за исключением алгоритма поиска 
            воробьев, показали результат лучше, чем случайный поиск;
            \item[б] Выбранные с помощью поиска голодных игр и алгоритма оптимизации кормодобывания 
            скатов манта параметры, по сравнению с параметрами по умолчанию, показали 
            либо такие же, либо на десятые доли процента более высокие метрики;
            \item[в] Выявлен наиболее эффективный в рамках данной задачи и выбранных метрик природный алгоритм — 
            алгоритм оптимизации кормодобывания скатов манта.
        \end{itemize}
\end{itemize}

В дальнейших исследованиях планируется применение полученных на данном этапе классификаторов к 
другим наборам данных с целью сравнения качества их обучения.


% — +Программный комплекс для проведения экспериментов 
% — +Набор экспериментальных метрик для каждого алгоритма, на основе которых можно выявить наиболее эффективный из предложенных алгоритмов
% — Выводы из сравнения алгоритмов
