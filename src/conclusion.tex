\chapter*{Заключение}\addcontentsline{toc}{chapter}{Заключение}

Фильтрация спама была рассмотрена в контексте машинного обучения как задача
классификации. Описаны некоторые алгоритмы машинного обучения, используемые 
для решения данной задачи, а именно: наивный байесовский и мультиномиальный байесовский 
классификаторы, метод опорных векторов и метод опорных векторов со стохастическим градиентным спуском. 

Приведены метрики, используемые для оценки производительности
моделей машинного обучения: Accuracy, Recall, Precision, F1-мера, кривая ошибок и площадь под кривой ошибок. 

В работе перечислены основные подходы к решению задачи оптимизации параметров
обучающих алгоритмов, такие как поиск по сетке, случайный поиск, градиентный спуск, байесовская 
оптимизации и эвристический подход, к которому относится применение природных алгоритмов.
В рамках данной задачи спланирован и проведен эксперимент по сравнению 
эффективности нескольких природных алгоритмов: оптимизатор орла (AO), поиск голодных игр (HGS), алгоритм поиска 
воробьев (SSA) и алгоритм оптимизации кормодобывания скатов манта (MRFO).

В ходе работы получены следующие результаты:
\begin{itemize}
    \item[—] Спроектирован программный комплекс для проведения эксперимента по сравнению эффективности 
        выбранных природных алгоритмов в задаче оптимизации классификатора спама на основе метода опорных векторов;
    \item[—] Выбраны метрики для оценки эффективности природных алгоритмов в 
        вышеуказанной задаче — $Accuracy$ и площадь под кривой ошибок; 
    \item[—] Создан программный комплекс для проведения эксперимента;
    \item[—] Проведен эксперимент, на основе выбранных метрик сделаны выводы об 
    эффективности выбранных алгоритмов:
        \begin{itemize}
            \item[а] Влияние настройки параметров на эффективность алгоритмов более заметно на меньших объемах данных. 
            При использовании для обучения большого набора данных, 
            алгоритмы оптимизации в большинстве случаев дают такое же улучшение, как и случайный поиск;
            
            \item[б] Алгоритмы SSA и AO, хоть и показали на новых данных в большинстве случаев 
            более высокую $Accuracy$ по сравнению со случайным поиском и параметрами по умолчанию, по показателю площади 
            под кривой ошибок $ROC$ оказались значительно хуже других алгоритмов;
            
            \item[в] "Поиск голодных игр" (HGS) позволил улучшить производительность по обеим метрикам в большем числе 
            случаев, чем у других выбранных алгоритмов;

            \item[г] "Алгоритм кормодобывания скатов" (MRFO) также показал хорошие результаты по обеим метрикам, но 
            несколько хуже, чем HGS.
        \end{itemize}
\end{itemize}

Таким образом, использование алгоритма "поиск голодных игр" или алгоритма кормодобывания скатов 
способно повысить точность определения нежелательных рассылок при автоматической фильтрации 
почтового трафика на величину от 0,1\% до 2\% по сравнению со случайным поиском и до 9\% по сравнению с 
параметрами по умолчанию.

