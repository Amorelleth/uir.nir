\section{Программная реализация эксперимента}

Основные этапы работы программы:

\begin{itemize}
    \item[—] Считывание и преобразование данных с помощью библиотек numpy и pandas;
    \item[—] Перед обучением или применением всех классификаторов осуществляется преобразование текста в 
    числовую матрицу слов с использованием \\TfidfVectorizer\cite{scikitTfIdf} и набора стоп-слов от модуля nltk;
    \item[—] Получение параметров для каждого классификатора методом случайного поиска с перекрестной 
    проверкой и с помощью био-алгоритмов с перекрестной проверкой (из 3 обучающих наборов и 6 методов оптимизации получено $3 \times 6 = 18$ классификаторов);
    Природные алгоритмы применяются с использованием модуля mealpy \cite{thieu_nguyen_2020_3711949};
    \item[—] Оценка точности классификаторов с помощью процедуры перекрестной проверки;
    \item[—] Оценка точности и площади под ROC-кривой классификаторов на двух не использовавшихся при обучении наборах данных;
    \item[—] Построение графиков ROC-кривых для всех комбинаций обучающего и тестового набора (всего 6 графиков).
\end{itemize}

\subsection{Настройка параметров модели}

Параметры настройки модели имеют большое влияние на обнаружение спам-писем и скорость обучения. Для алгоритма SGD 
подбирались следующие параметры:

\begin{itemize}
    \item[—] Альфа (alpha) — чем выше значение, тем сильнее регуляризация. Также может использоваться для вычисления скорости обучения;
    \item[—] Эпсилон (epsilon) — значение определяет скорость обучения алгоритма;
    \item[—] Тол (tol) — критерий остановки.
\end{itemize}

Настройка модели проводилась с использованием:

\begin{itemize}
    \item[—] Параметров по умолчанию;
    \item[—] Случайного поиска (подраздел \ref{optimization});
    \item[—] Природных алгоритмов (раздел \ref{BIOAlgs}).
\end{itemize}

Границы для всех трех параметров были установлены $10^{-3}$ до $10^3$ с шагом в одну степень.
Была применена стратифицированная перекрестная проверка с числом разбиений, равным 10.

Недостатки использования случайного поиска и поиска по сетке очевидны — и тот, и другой используют и возвращают только те
параметры, которые им переданы в качестве словаря, с той лишь разницей, что случайный поиск не предполагает полного перебора,
а потому быстрее, чем поиск по сетке.

\subsection{Применение природных алгоритмов оптимизации}

Для поиска внутри области, а не по точкам, можно использовать природные алгоритмы.

В контексте данной работы задача оптимизации — максимизировать метрическую функцию (раздел \ref{Section:Performance}).
В качестве такой функции была использована метрика $Accuracy$.

Алгоритмы оптимизации были запущены с использованием стратифицированной перекрестной проверки с разбиением на 10 частей.
Число частиц было принято равным 10, число популяций — равным 40.
Границы поиска для каждого параметра — от $10^{-3}$ до $10^3$.
