\chapter*{Реферат}
\thispagestyle{plain}
Пояснительная записка содержит 60 страниц, 6 рисунков, 2 приложения и 25 источников.

Ключевые слова: СПАМ, МАШИННОЕ ОБУЧЕНИЕ, МЕТОД ОПОРНЫХ ВЕКТОРОВ, ОПТИМИЗАЦИЯ, 
ПРИРОДНЫЙ АЛГОРИТМ, ГИПЕРПАРАМЕТР, ACCURACY, PRECISION, RECALL, F1-SCORE.

Целью работы является выявление наиболее эффективного нового природного алгоритма 
для классификации англоязычного спама на основе метода опорных векторов. 

Объект исследования — классификаторы спама на основе машинного обучения. 

Предмет исследования — оптимизация гиперпараметров алгоритмов машинного обучения.

Методы исследования — анализ существующих подходов к оптимизации гиперпараметров, 
создание системы подбора гиперпараметров с применением природных алгоритмов.

Результатом данной работы является получение экспериментальных оценок новых 
природных алгоритмов, а также выявление наиболее эффективного из них 
в рамках данной задачи.

Качественный подбор параметров алгоритма машинного обучения является актуальной задачей, 
поскольку они во многом определяют точность работы получившегося спам-классификатора.