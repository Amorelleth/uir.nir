\chapter*{Определения, обозначения и сокращения}\addcontentsline{toc}{chapter}{Определения, обозначения, сокращения}

В настоящем отчете применяются следующие термины с соответствующими определениями, 
обозначениями и сокращениями:

\begingroup
\setlength{\baselineskip}{1.5em}
\renewcommand{\arraystretch}{1.5}
\begin{table}[ht]
    % \centering
    \begin{tabular}{ p{0.1\textwidth} p{0.03\textwidth} p{0.8\textwidth} } 
        Спам & — & нежелательные сообщения, массово рассылаемые по электронной почте \\
        МО & — & машинное обучение \\
        MTA & — & агент передачи сообщений, программное обеспечение, которое передает сообщения электронной почты с одного компьютера на другой с помощью SMTP (Message Transfer Agent) \\
        SI & — & роевой интеллект, коллективное поведение 
        децентрализованных, самоорганизованных систем, естественных или 
        искусственных (Swarm Intelligence) \\
        AO & — & Оптимизатор орла (Aquila optimizer) \\
        HGS & — & Поиск голодных игр (Hunger Games Search) \\
        SSA & — & Алгоритм поиска воробьев (Sparrow Search Algorithm) \\
        MRFO & — & Алгоритм оптимизации кормодобывания скатов манта (Manta Ray Foraging Optimization) \\
        SVM & — & Метод опорных векторов (Support vector machine) \\
        SGD & — & Стохастический градиентный спуск (Stochastic gradient descent) \\
    \end{tabular} 
\end{table}
\endgroup