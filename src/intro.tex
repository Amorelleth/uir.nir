\chapter*{Введение}\addcontentsline{toc}{chapter}{Введение}

Электронная почта упростила способы общения как для многих организаций, так и для
частных лиц. Однако этот способ общения часто используется спамерами для получения
мошеннической выгоды путем рассылки нежелательных электронных писем. \cite{IEEE}
По данным Лаборатории Касперского за 2021 год, 45,56\% электронных писем было спамом, 
причем большая его часть (24,77\%), как и в прошлом году, исходила из России \cite{stat}. На этом фоне
более быстрое и точное детектирование спама остается актуальной задачей.

Для решения задачи фильтрации нежелательных рассылок часто применяются
алгоритмы машинного обучения. Качество их работы во многом зависит от правильно
подобранных параметров, которые необходимо подбирать экспериментальным путем.
Зачастую это осуществляется методами перебора, что не всегда дает желаемую
производительность. В связи с этим применяются алгоритмы оптимизации, в том числе
и природные.

В течение сотен лет эволюции многие живые организмы развили особые способности,
не дающие им погибнуть. Успех биологических организмов вдохновил исследователей,
занимающихся задачами оптимизации, на создание алгоритмов,
копирующих поведение природы. За последние годы сообщество таких исследователей
значительно выросло, достигнув большого разнообразия в том, что касается их
источников вдохновения \cite{BioInspiredTaxonomy}.

Современными исследованиями в области природных алгоритмов занимаются такие учёные, как Костенко В. А., 
Peter J. Bentley, Derviş Karaboğa, Xin-She Yang.

Данная работа, в соответствии с федеральным государственным образовательным стандартом высшего профессионального
образования по направлению подготовки 090900 «Информационная безопасность», отвечает следующим задачам
экспериментально-исследовательской деятельности: сбор, изучение научно-технической информации, отечественного и
зарубежного опыта по тематике исследования; проведение экспериментов по заданной методике, обработка и анализ результатов.

Все представленные в работе результаты получены автором лично.

В первом разделе проводится обзор различных подходов к фильтрации спама.

Во втором разделе фильтрация спама рассматривается как задача машинного обучения,
приводятся некоторые алгоритмы машинного обучения, используемые для ее решения.

В третьем разделе проводится обзор существующих подходов к оптимизации алгоритмов машинного обучения, 
в том числе природных алгоритмов.

В четвертом разделе рассмотрены метрики, используемые для оценки производительности
моделей машинного обучения.

В пятом разделе приведен план эксперимента, описана система для проведения эксперимента.

В шестом разделе описаны использованные в данной работе природные алгоритмы.

В седьмом разделе определены инструменты для программной реализации системы.

Восьмой раздел содержит описание реализованной системы.

В девятом разделе приведены результаты экспериментальной оценки эффективности 
выбранных природных алгоритмов, сделаны соответствующие выводы.

