\section{Определение набора алгоритмов для проведения эксперимента}\label{BIOAlgs}

Для данной работы решено было использовать алгоритмы роевого интеллекта, опубликованные в 2020-2021 годах:

\begin{itemize}
    \item[—] AO — Aquila Optimizer;
    \item[—] HGS — Hunger Games Search;
    \item[—] SSA — Sparrow Search Algorithm;
    \item[—] MRFO — Manta Ray Foraging Optimization.
\end{itemize}

\subsection{Оптимизатор орла (Aquila optimizer)}\label{AO}

В работе \cite{AO} представлен новый метод оптимизации на основе популяций, называемый оптимизатор орла,
основанный на поведении орлов в природе во время охоты. Оптимизационные функции
данного алгоритма обусловлены четырьмя процессами: выбор пространства поиска высоким
взлетом с вертикальным наклоном, разведка в пределах расходящегося пространства поиска
контурным полетом с короткой планирующей атакой, использование в пределах сходящегося пространства
поиска низким полетом с атакой медленного спуска, а также бег и захват добычи.

\subsection{Поиск голодных игр (Hunger Games Search)}\label{HGS}

Метод оптимизации на основе популяций, называемый поиском голодных игр, основан на поведенческом выборе
животных в соответствии с чувством голода. Этот метод поиска использует простую концепцию голода как наиболее
сильную мотивацию и причины поведения всех живых существ. Кроме того, почти все животные используют 
вычислительно-лог\-ические правила (игры). Часто такие действия являются следствиями эволюции.

Особенностью данного метода, по словам авторов, является его динамический характер, простая структура и
высокая производительность, гибкость и масштабируемость, а также тот факт, что метод был не только
проверен на хорошо известном наборе тестовых функций, но и применен к нескольким инженерным задачам \cite{HGS}.

\subsection{Алгоритм поиска воробьев (Sparrow Search Algorithm)}\label{SSA}

Алгоритм поиска воробьев представляет собой новый подход к оптимизации на основе роевого интеллекта.
Метод вдохновлен поведением воробьев в поисках пищи и в борьбе с хищниками. Авторы алгоритма
утверждают, что предлагаемый метод может обеспечить высококонкурентные результаты. Более того, результаты
двух практических инженерных задач также показывают, что алгоритм имеет высокую производительность в различных
областях поиска \cite{SSA}.

\subsection{Алгоритм оптимизации кормодобывания скатов манта (Manta Ray Foraging Optimization)}\label{MRFO}

В основе природного алгоритма оптимизации кормодобывания скатами манта (MRFO) лежит разумное поведение
скатов-мантов. Целью данного алгоритма, по утверждению авторов, является обеспечение альтернативного подхода
к оптимизации для решения реальных инженерных проблем. Производительность алгоритма была оценена
путем сравнения с другими современными оптимизаторами, с помощью функций оптимизации тестов и восьми
реальных примеров инженерного проектирования. Результаты сравнения тестовых функций показывают, что данный подход
намного превосходит своих конкурентов. Кроме того, реальные инженерные приложения демонстрируют достоинства
этого алгоритма в решении сложных проблем с точки зрения затрат на вычисления и точности решения \cite{MRFO}.

